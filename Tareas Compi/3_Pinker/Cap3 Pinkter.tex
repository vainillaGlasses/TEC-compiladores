\documentclass[13pt,letterpaper,onecolumn]{report}
\usepackage[utf8]{inputenc}
\usepackage{amsmath}
\usepackage{amsfonts}
\usepackage{amssymb}
\usepackage{graphicx}
\graphicspath{ {img/} }


\usepackage{geometry}
\geometry{letterpaper, margin=1in}



\author{Samantha Arburola Leon}
\title{The Language Instinct}
\date{\today}

\begin{document}
%title page
\pagenumbering{gobble}
\vspace{10em}
\begin{center}
\includegraphics{tec} \\
\vspace{2em}
Computer Science School\\Compilers and Interpreters\\
\vspace{1em}
Professor\\Dr.Francisco Torres Rojas \\
\vspace{7em}
Summary \\ The Language Instinct\\Steven Pinker\\Chapter 3\\
\vspace{5em}
Student\\ Samantha Arburola Le\'on \\ 2013101697\\
\vspace{10em}
March 1rst, 2017
\end{center}
\pagebreak[4]

%summary chapter 1
\pagenumbering{arabic}
\begin{center}
{\huge\bfseries 3\\Mentalese\par} \vspace{2em}
\end{center}
\hspace{1em}For 2050 the ultimate technology for thought control would be in place: the language Newspeak. In an appendix to Nineteen Eighty-four Orwell wrote:\\
\begin{quote}
purpose of Newspeak was not only to provide a medium of expression for the world-view and mental habits proper to the devotees of Insoc (English Socialism), but to make all other modes of thought impossible ...This was done partly by invention of new words, but chiefly by eliminating undesirable words and by stripping such words as remained of unorthodox meanings, and so far as possible of all secondary meanings whatever ...A person growing up with Newspeak as his language would no more know that equal had once had once meaning of \textit{politically equal} or that free had once meant \textit{intellectually free}
\end{quote} 
It leave a door open for injuries never defined or know.\\

\hspace{1em}\textbf{Do people literally think in English(or another language) or by 2050 in Newspeak?} No, there’s a language of thought or mentalese, and about that this chapter talk about. Philosophers argue that since animals lack language, they must also lack consciousness, and therefore they do not possess the rights of conscious beings. General Semantics lay the blame for human folly on insidious “semantic damage” to thought perpetrated by the structure of language.\\

\hspace{1em}Sapir-Whorf hypothesis of linguistic determinism, stating that people’s thoughts area determined by categories made available by their language and linguistic relativity differences among languages cause differences in the thoughts of their speaks. The fundamentally Hopi concept of time ot the dozens of Eskimo words for snow. The implications is heavy: the foundational categories of reality are not “in” the world but are imposed by one’s culture. But is wrong. We have all had the experience of uttering or writing a sentence, then stopping and realizing that it wasn’t exactly what we mean to say. Sometimes it is not easy to find any words that properly convey a thought. What we meant to say is different from what we said.\\

\hspace{1em}Language must affect thought. For example “revenue enhancement” has a much broader meaning than “taxes” and listeners naturally assume that if a politician had meant taxes he would have said taxes, but a point out is that people are not so brainwashed. There is less of a temptation to equate it with language just because words are more palpable than thoughts.
\\

\textbf{Linguistic determinism hypothesis}\\
\hspace{1em}Franz Boas and his students argued that nonindustrial people were not primitive savages but had systems of languages, knowledge and culture as complex and valid in their world view as our own, when English speakers decide whether or not to put -ed onto the end of a verb, they must pay attention to tense, but when they decide which suffix to put on their verbs, they must pay attention to whether the knowledge, they are conveying was learned through direct observation or by hearsay.\\

\hspace{1em}How language led workers to misconstrue dangerous situations depends on manner how workers understand what their boss said. The more examine Whorf’s arguments, the less sense they make. Surely this walking catastrophe was fooled by an action and not by the language. Whorf presented many such examples from Native American languages. The Apache equivalent of “He invites people to feast” become “He, or somebody, goes for eaters of cooked food” is utterly unlike our way of talking.\\
His assertions about Apache psychology are based entirely on Apache grammar, they speak differently, so they must think differently. You can know that they think differently just listening to the way they speak. Meaning change too much when Whorf rendered sentences word-for-word as mechanical and not on an expressive way.\\

\hspace{1em}How strange a language mind must be is cause by color on words. Colors are colors for everyone, but how much mean is different as we all are humans but we are different so a variety of meanings are given to same concept. The way we see colors determines how we learn words for them, not vice versa, and focus on change and process itself and on psychological distinctions between presently know, mythical and conjecturally distant. The Hopi also had little interest in “exact sequences, dating, calendars, chronology”. Hopi speech contains tense, metaphors for time units for time like “ancient, quick, long time and finished”. Contrary to popular belief the Eskimos do not have more words for snow than do speakers of English. They do not have four hundred words for snow, as it has been claimed in print, one dictionary puts the figure at two. Where did the myth come from? The anthropologist Laura Martin has documented how the story grew like an urban legend, exaggerated with each retelling.\\

\hspace{1em}The linguist Geoffrey Pullum speculates: 
\begin{quote}
The alleged lexical extravagance of the Eskimos comports so well with the many other facets of their polysynthetic perversity.
\end{quote}
But the supposedly mind-broadening anecdotes owe their appeal to a patronizing willingness to treat other cultures’ psychologies as weird and exotic compared to our own.\\

\hspace{1em}In a typical experiment, subjects have to commit paint chips to memory and are tested with a multiple-choice procedure, the subjects show slightly better memory for colors that have readily available names in their languages. But even colors without names are remembered fairly well, so the experiment does not show that the colors are remembered by verbal labels alone. All it shows is that the colors are remembered the chips in two forms, a nonverbal visual image and the verbal label, presumably because two kinds of memory. Boom concluded that the Chinese language renders it speakers unable to entertain hypothetical false worlds without great mental effort. The cognitive psychologist Terry Au, Yohtaro Takano and Lisa Liu, identified serious flaws in BLoom’s experiments, his stories turned out, upon careful rereading, to be genuinely ambiguous. Chinese college students tend to have more science training than American students, and thus they were better at detecting the ambiguities that Bloom himself missed.\\

\hspace{1em}To know what someone else is thinking or to talk to each other about the nature of thinking, we have to use ¡words! As a cognitive scientist I can afford to be smug about common sense being true and linguistic determinism being a conventional absurdity. Ingenious examples of teach them language involves: \\
\textbf{first babies}: who cannot think in words because they have not yet learned any. That five-month-old babies can do simple form of mental arithmetic. The methodology has shown that babies as young as five days old are sensitive to number. If there’s more objects the surprised babies stare longers.\\
\textbf{second monkeys}: who cannot think in words because they are incapable of learning them. The monkeys appeared to be relying not only on physical resemblance between a given pair of monkeys, or on the sheer number of hours they had previously spent together, but something more subtle in the history of their interaction. \\
\textbf{third human adults}: who whether or not they think in words, claim their best think is done without them. Many creative people insist that in their most inspired moments they think not in words but in mental images. Physical scientists are even more adamant that their thinking is geometrical, not verbal.\\

\hspace{1em}The most famous self-described visual thinker is Albert Einstein, who arrived at some of his insights by imagining himself riding a beam of light and looking back at clock or dropping a coin while standing in a plummeting elevator. He wrote: “...Conventional words or others signs have to be sought for laboriously only in a secondary state, when the mentioned associative play is sufficiently established and can reproduced at will.\\

\hspace{1em}Another creative scientist, Roger Shepard, had his own moment of sudden visual inspiration and it led to a classic laboratory demonstrations of mental imagery in mere mortals.In Shepard experience “a spontaneous kinetic image of three-dimensional structures majestically turning in space”. In other words, the father the subjects had to mentally rotate a letter(in this experiment) the longer they took, from that they estimate that letters resolve in the mind at a rate of 56RPM. Many other experiments have corroborated the idea that visual thinking uses not language but a mental graphics system, with operations that rotate, scan, zoom, pan displace and fill in patterns of contours.\\

\hspace{1em}\textbf{What sense, then, can we make of the suggestions that images, numbers, kinship relations or logic can be represented in the brain without being couched in words?} Alan Turing, the brilliant British mathematician and philosopher, to make the idea of a mental representation scientifical respectable. Turing described a hypothetical machine that could be said to engage in reasoning. In fact this simple device, named a Turing Machine in his honor, is powerful enough to solve any problem that any computer, past, present or future can solve. By looking at how a Turing machine works, we can get a grasp of what it would mean for a human mind to think in mentalese as opposed to English.\\
A simple example is the old chestnut from introductory logic: if you know that Socrates is a man, and all men are mortal, you can figure out that Socrates is mortal. The first key idea is representation: a physical object whose parts and arrangement correspond piece to some set of ideas or facts. To get reasoning to happen we now need a processor: can react to different pieces of a representation and do something in response, including altering the representation or making new ones. What makes the whole device smarts is the exact correspondence between the logician’s rules and the way the device scans, moves and prints.\\Turing showed the machine can do anything that any computer can do, and he conjectured, anything that any physical embodied mind can do.\\

\hspace{1em}Remember that a representation does not have to look like English or any language, it just has to use symbols to represent concepts and arrangements of symbols to represent the logical relations among them. Do they fact? Clearly no. Any language people speaks is hopelessly unsuited to serve as our internal medium of computation.\\ People do not think in English or Chinese or Apache, they think in a language of thought, This language of thought probably looks a bit like all these languages, presumably it has symbols and arrangements of symbols. Mentalese must be simpler than spoken languages; conversation-specific words and constructions are absent and information about pronouncing words or even ordering them is unnecessary, and that is a universal mentalese.


\end{document}