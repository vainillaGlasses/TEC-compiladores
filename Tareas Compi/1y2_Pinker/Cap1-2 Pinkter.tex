\documentclass[13pt,letterpaper,onecolumn]{report}
\usepackage[utf8]{inputenc}
\usepackage{amsmath}
\usepackage{amsfonts}
\usepackage{amssymb}
\usepackage{graphicx}
\graphicspath{ {img/} }

\author{Samantha Arburola Leon}
\title{The Language Instinct}
\date{\today}

\begin{document}
%title page
\pagenumbering{gobble}
\vspace{10em}
\begin{center}
\includegraphics{tec}
Computer Science School\\Compilers and Interpreters\\
\vspace{1em}
Professor\\Dr.Francisco Torres Rojas \\
\vspace{7em}
Summary \\ The Language Instinct\\Steven Pinker\\Chapter 1 and 2\\
\vspace{5em}
Student\\ Samantha Arburola Le\'on \\ 2013101697\\
\vspace{10em}
February 17th, 2017
\end{center}
\pagebreak[4]

%summary chapter 1
\pagenumbering{arabic}
\chapter{An Instinct to Acquire an Art}
\hspace{1em}Language is that especial thing which make a humans different from the rest of living creatures.As the author said it give to us the "ability of shape events in each other's brains with exquisite precision". Writing is a way for communication but it is not good as spoke, however well written and used all available symbologies can never be equaled verbal communication. The language first existed spoken and then looked for the way to represent it; even there are aboriginal dialects that do not have writing.\\

\hspace{1em}Learn, speak and understand language is an  instinct included in everyone nature. Nowadays the cognitive science allows the explanation about how human intelligence works. There are different ways for teach children learn to talk in school can be found grammatical sophistication but sagging education standards led persons to decline the construction of grammatical sentences but even with greater heights of spelling there's a spell-it-like-it-sound system which must be prevented. Language is a complex, specialized skill started in the develop spontaneously biological composition of children's brains. So it is described as instinct by the cognitive  scientists who define it as a psychological faculty, a mental organ, a neural system and a computational module.\\

\hspace{1em}Language considered as a instinct leave the concept of a cultural invention and converts it in a ability, this point of view reveals a biological facts instead a cultural intervention because species can act by their own to refine details in the language taking qualities of the environment and healthy circumstances.\\
\\Pinker voice:
\begin{verse}
“A preschooler's tacit knowledge of grammar is more sophisticated than the thickest style manual of the most state-of-the-art computer language system”
\end{verse}
Giving a notorious axiom of the complexity of the language punctuated by scientists and of which a child is not conscious to be this something so natural.\\

\hspace{1em}The first one in said the language is a instinct was Dawin in 1871 pleading language is an art and a tendency to speak. There’s the supposition that language; any of them, has been deliberately invented by many steps developed unconsciously. From all of this Darwin concluded that language is an instinctive tendency sprout by the brutal human being born in the flexible interplay need of communication and not a elegant and intellectual act.\\

\hspace{1em}From the look it gives to the animal kingdom, it is easy to associate how we obey every impulse guided by the instinct make each living being auto sufficient in the moment when should behaivor in proper thing to do or be.\\

\hspace{1em}A richness and beauty system is hided by idea of the human need of learn to speak imitating the movements and sounds of the mother when she is teaching to talk, but there’s more than a learing, it is a natural gift which appears transparently  and automatically.\\

\hspace{1em} Chomsky called attention to the fundamental facts about how the child adopts the syntactic patterns out of speech of their parents and the brain capacity of making a infinite number of sentences with short list of words. Language is not a physical organ but it became in a universe with a complex construction in grammar and meaning; taking count how many language can be and the characteristics that make same language change their identity depending who speaks, and there is where culture and knowledge play a role because the mind is so extraordinary that make available the interchange of ideas.\\

\hspace{1em}The language is consider as an evolutionary adaptation for analysis the many arguments that converge in biological need of share and express of the human being what it feels and learn from experience.

\pagebreak[4]

%summary chaper 2
\chapter{Chatterboxes}
\hspace{4em} The language gives identity to society. The universality of complex language is a discovery that fills linguists with awe because culture is not the reason of appearance of language. Lots of knowledge in lexicon requires study but express through words is a special instinct. Actually, the people whose linguistic abilities are most badly underestimated are right here in our society.\\

\hspace{1em} The best definition comes from Max Weinreich which said : “a language is a dialect with army and a navy”.  In the text an interview appears write, and allow you to notice a big different of the book redaction and lexicon compared with Larry’s. It open to you a way to experience the dialect of gang members. From there rules of grammar and lexicon for expression vary depends of speaker.\\

\hspace{1em} Not everything that is universal is innate. If eats is so natural, a special hand-to-mouth instinct so difficult to explain for do it with feet. The language is limited by the human capability of information processing, but a strategy would be learn for increase the flexible human intelligence for assimilate a complex language universe. Even specific ethnicities accept mixtures when speakers communicate and time also is an influential axiom but no enough strong has geographic interventions in pronunciation and sentences grammar construction. The people grammar who is exposed to the pidgin shows up errors make when acquiring a established and embellished language.\\
\hspace{1em}The language is a collective product born of minds that increments grammatical complexity to received message in communication process. When deaf infants learns spoken sing language, more of them have not access to it and they are force to use a oralist tradition by educators who want to force them to master lip reading and speech. A difficult intellectual puzzle is lived with deaf adults in a foreign language class more than with deaf children who takes advantage of media communications. In the sign  language exist a little modifications but preserving meaning. A example of creolization by a single living child is when  child surpass his parents by superimposing verb inflection and reinvented the system onto single verb in a specific order.\\
\hspace{1em} Children deserve most of the credit for the language they acquire because folklore of parents can intervene but a grammatical world is build in their minds as algorithms on labels for subjects and verbs. Some could be wrong but as linear rules, we must understand the structure-sensitive rule by proper wording capability.\\
                   
\hspace{1em} The language gives identity to society. The universality of complex language is a discovery that fills linguists with awe because culture is not the reason of appearance of language. Lots of knowledge in lexicon requires study but express through words is a special instinct. Actually, the people whose linguistic abilities are most badly underestimated are right here in our society.\\

\hspace{1em} The best definition comes from Max Weinreich which said : “a language is a dialect with army and a navy”. In the text an interview appears write, and allow you to notice a big different of the book redaction and lexicon compared with Larry’s. It open to you a way to experience the dialect of gang members. From there rules of grammar and lexicon for expression vary depends of speaker.\\

\hspace{1em} Chomsky’s claim was tested in an experiment at a daycare center. The children cheerfully provided the appropriate questions, and, as Chomsky would have predicted, not a single one of them came up with an ungrammatical string. The universal constraints on grammatical rules also show that the basic form of language can not be explained away as the inevitable outcome of a drive of usefulness. Evidence corroborating the claim that the mind contains blueprints for grammatical rules come out of mouths of babes and suchlings.\\

\hspace{1em} Around of the three years children start exploring their grammar explosion on the fine points of their community skills on spoken language. When a child start to add the -s on verbs which never listen from parents is the moment that language acquisition cannot be explained as a kind of imitation.\\

\hspace{1em} If language is an instinct, it should have an identifiable seat in the brain, and even a special set of genes that help wire it into place; here in this point is where I think we can prove that all those manners that we can compare between cousins or grandparents and grandchildren are scientifically true, not only are inheritance of physical abilities and qualities, the popular soft skills too. Sometimes we say “he speaks on same way as his father”, there’s a big possibility that not only inheritance the manner also the language by the genetics.\\

\hspace{1em} When we speak the communication will be better in expressions and how the others catch the idea, but some endings could be lost the opposite happen  when writing is done,  It's worth saving that everything must be correctly executed.\\

\hspace{1em} Intellectual functions are not closely tied to language, such as knowledge of right and left, ability to draw, to calculate or carry on commands. A healthy children can develop language on schedule, the ones who need therapists help, where the language-impaired children were randomly distributed among different families, sexes and birth orders. What does this hypothetical gene do? It is their language that is impaired, but they are not like Broca’s aphasics is more like speak with a foreign, who carefully plan what to say in slowly and collaboratively communicates with the citizen who has greater skills on the language.\\
\hspace{1em} Diseases or congenital problems do not avoid people to communicate, some impaired like Broca’s aphasia and SLI are cases where language demands full capacity for develop it but do not affect the intelligence of the persona, and when them get the control of the right communication skills, the ability can flourish because the mind is in all capability. The case of Deyse, she was born with spina bifida, hydrocephalic children occasionally end up like she, significantly retarded but with unimpaired language skills as the patients with Williams syndrome who understand complex sentences and fix ungrammatical sentences at normal levels.\\

\hspace{1em} In conclusion we see that you do not intellectual wherewithal to function in society, particularly skills or advantages, a language user just need the right genes or bits of brain for share the expression by language.

\end{document}